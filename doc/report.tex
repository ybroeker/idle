\documentclass[a4paper,10pt]{scrartcl}
\usepackage[utf8]{inputenc}
\usepackage{times}
\usepackage[ngerman]{babel}
\usepackage{amsmath}
\usepackage{amsfonts}
\usepackage{amssymb}
\usepackage{fancyhdr}
\usepackage{graphicx}
\usepackage{listings}
\usepackage{pdfpages} 
\usepackage{listings}
\usepackage[lmargin=2cm,rmargin=2cm,tmargin=2cm,bmargin=2cm]{geometry}

\inputencoding{utf8}
\parindent0em
\pagestyle{fancy}

%opening
\fancyhf{}
%%Titel der Veranstaltung -- Ersetze X durch Zettelnr.
\fancyhead[L]{\textbf{}}
%%Dein Name
\fancyhead[C]{\textbf{}}
%%Dein Tutor
\fancyhead[R]{\textbf{}}
%%\fancyfoot[C]{\thepage}

\newcommand{\M}{\begin{pmatrix}}
\newcommand{\m}{\end{pmatrix}}
\newcommand{\Pt}{\begin{Vmatrix}}
\newcommand{\pt}{\end{Vmatrix}}
           
        
        
\newcommand{\e}{ \mathrm e}
\renewcommand{\i}{ \mathrm i}
\newcommand{\arsinh}{\mathrm{arsinh}}
\newcommand{\rang}{\mathrm{rang}}
\newcommand{\zz}{\mathrm{\hbox{Z}\kern-.4em\raise-0.4ex\hbox{Z}} \hbox{: } }
\newcommand{\mb}[1]{\mathbb{#1}}

\let\stdsubsection\subsection
\renewcommand\subsection{\nopagebreak\stdsubsection}

\begin{document}

\tableofcontents

\section{how to run}

\subsection{Animationenginekonfiguration?}

\subsection{gesturebinding}

\subsection{BML-Aufruf}

\begin{lstlisting}[language=XML]
<bml id="bml1" xmlns="http://www.bml-initiative.org/bml/bml-1.0">
    <postureShift id="pose1" start="0">
        <stance type="STANDING"/>
        <pose part="BODY" lexeme="IDLE"/>
    </postureShift>
</bml>
 \end{lstlisting}

\subsubsection{Parameter}
Klassen? Mocaps?


\section{Description of used techniques with references to literature}

\subsection{MotionGraphs}
\subsection{Distance Metrics}
\subsection{Blending}
\subsection{Align, Split, etc}

\section{System overview (e.g class structure / functional overview / dataflow)}

\subsection{Class structure}
\subsubsection{MotionGraph}
\subsubsection{IDistance}
\subsubsection{JointAngles}
\subsubsection{IBlend}
\subsubsection{Blend}
\subsubsection{IAlignment}
\subsubsection{Alignment}
\subsubsection{ISplit}
\subsubsection{DefaultSplit}
\subsubsection{IdleMovement}
\subsubsection{MotionGraphBuilder}


\subsection{Startup}
Builder/MotionGraph.init()

\subsection{play()}

\section{Extensibility (e.g. How to use with other mocap files / visemes)}

\subsection{Load mocap}

\subsection{Other Classes?}

%\includepdf[pages={-}]{ueb12.pdf}
\end{document}